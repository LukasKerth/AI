% this template is an adaption of:
% https://www.overleaf.com/latex/templates/template-for-theoretische-informatik-uni-tubingen/xwsycshfkjtf


% change for each exercise
\newcommand{\NUMBER}{1}
\newcommand{\EXERCISES}{3}

% only once
\newcommand{\COURSE}{Artificial Intelligence, Winter Semester \the\year{}}
\newcommand{\STUDENTA}{Lukas Kerth}
\newcommand{\STUDENTB}{Jonas Mohr}


% keep as is --------------------------------------------------------
\documentclass[a4paper]{scrartcl}
\usepackage[utf8]{inputenc}
\usepackage[english]{babel}
\usepackage{amsmath}
\usepackage{amssymb}
\usepackage{fancyhdr}
\usepackage{color}
\usepackage[table,xcdraw]{xcolor}
\usepackage{graphicx}
\usepackage{lastpage}
\usepackage{listings}
\usepackage{tikz}
\usepackage{pdflscape}
\usepackage{subfigure}
\usepackage{float}
\usepackage{polynom}
\usepackage{hyperref}
\usepackage{tabularx}
\usepackage{forloop}
\usepackage{geometry}
\usepackage{listings}
\usepackage{fancybox}
\usepackage{tikz}
\usepackage{algpseudocode,algorithm,algorithmicx}

\input kvmacros
\geometry{a4paper,left=3cm, right=3cm, top=3cm, bottom=3cm}

% headline
\def\header#1{
	\begin{center}
		{\Large Assignment Sheet #1}
	\end{center}
}

% point table
\newcounter{punktelistectr}
\newcounter{punkte}
\newcommand{\punkteliste}[2]{%
	\setcounter{punkte}{#2}%
	\addtocounter{punkte}{-#1}%
	\stepcounter{punkte}%<-- also punkte = m-n+1 = Anzahl Spalten[1]
	\begin{center}%
		\begin{tabularx}{\linewidth}[]{@{}*{\thepunkte}{>{\centering\arraybackslash} X|}@{}>{\centering\arraybackslash}X}
			\forloop{punktelistectr}{#1}{\value{punktelistectr} < #2 } %
			{%
				\thepunktelistectr &
			}
			#2 &  $\Sigma$ \\
			\hline
			\forloop{punktelistectr}{#1}{\value{punktelistectr} < #2 } %
			{%
				&
			} &\\
			\forloop{punktelistectr}{#1}{\value{punktelistectr} < #2 } %
			{%
				&
			} &\\
		\end{tabularx}
	\end{center}
}

% header / footer
\pagestyle {fancy}
\fancyhead[L]{\COURSE}
\fancyhead[C]{}
\fancyhead[R]{\today}

\fancyfoot[L]{}
\fancyfoot[C]{}
\fancyfoot[R]{Page \thepage /\pageref*{LastPage}}

\begin{document}
	
% don't change
\begin{tabularx}{\linewidth}{m{0.2 \linewidth}X}
	\begin{minipage}{\linewidth}
		\STUDENTA\\
		\STUDENTB
	\end{minipage} & \begin{minipage}{\linewidth}
		\punkteliste{1}{\EXERCISES}
	\end{minipage}\\
\end{tabularx}

\header{Nr. \NUMBER}
% -------------------------------------------------------------------
	

% your solution


\section*{Question 1}

	\subsection*{a)}
% Please add the following required packages to your document preamble:
% \usepackage[table,xcdraw]{xcolor}
% Beamer presentation requires \usepackage{colortbl} instead of \usepackage[table,xcdraw]{xcolor}
\begin{table}[H]
	\begin{tabular}{|l|l|l|l|l|l|}
	\hline
	\rowcolor[HTML]{C0C0C0} 
	{\color[HTML]{000000} Enviroment} &
	  {\color[HTML]{000000} Observable} &
	  {\color[HTML]{000000} Deterministic} &
	  {\color[HTML]{000000} Static} &
	  {\color[HTML]{000000} Discrete} &
	  {\color[HTML]{000000} Number Agents} \\ \hline
	\cellcolor[HTML]{EFEFEF}Tennis           & fully  & deterministic & dynamic & continuous & multi  \\ \hline
	\cellcolor[HTML]{EFEFEF}Weather Forecast & partly & stochastic    & dynamic & continuous & single \\ \hline
	\cellcolor[HTML]{EFEFEF}Tic-Tac-Toe      & fully  & deterministic & semi    & discrete   & multi  \\ \hline
	\cellcolor[HTML]{EFEFEF}Cake             & fully  & deterministic & static  & discrete   & single \\ \hline
	\end{tabular}
	\end{table}
	\begin{itemize}
		\item Tennis $\rightarrow$ When playing Tennis, u play against someone.
		\item Weather Forecast $\rightarrow$ Weather forecast doesn't need a
		second agent to forecast the weather out of the given information.
		\item Tic-Tac-Toe $\rightarrow$ For playing tictactoe, there are 2
		players needed.
		\item Cake $\rightarrow$ Baking a cake can be done alone. 
	\end{itemize}
\section*{Question 2}
	\begin{itemize}
		\item False: Just because the agent doens't know everything, doesn't
		mean that the agent isn't rational. The agent can still be rational with
		given information like in a card game.
		\item False: A agent as a autonomous taxi driver cannot be perfectly
		rational when doing the task 'baking' because the PEAS is completly
		different. 
		\item False: The agent can loose if the opponent has a better hand.
		\item True: Every action performed, doesn't matter what the actions do,
		will be rewarded. 
	\end{itemize}
\end{document}

